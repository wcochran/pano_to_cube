\documentclass[11pt, oneside]{article}   	% use "amsart" instead of "article" for AMSLaTeX format
\usepackage{fullpage}
\usepackage{graphicx}
\usepackage{amssymb}

\title{Brief Article}
\author{The Author}
%\date{}							% Activate to display a given date or no date

\begin{document}
%\maketitle
%\section{}
%\subsection{}

\begin{center}
\includegraphics[width=0.6\textwidth]{cube_cylinder} 
\end{center}   
   
Given the cylinder $x^2 + y^2 = 1$ inscribed inside a $2 \times 2 \times 2$ cube
centered at the origin, we can project the cylindrical panorama onto the coresponding
four faces of the cube as follows.
For every pixel in the output image we cast a ray from the cube's center at $(0,0,0)$
through the corresponding point on one of the cube's faces and determine where that
ray intersects the cylinder. We then map this point on the cylinder to the corresponding
point in the input image and (bilinearly) sample the input image at the point. 
The resulting pixel color is assigned to the current output pixel.

We parameterize the ray emanating from 
the cube's center $(0,0,0)$ through a point $(u,v,w)$ on a face of the cube as
\begin{equation}
  \mathbf{r}(t) = (0,0,0) + t\cdot(u,v,w). \label{eq:r}
\end{equation}
This ray intersects the cylinder where
\begin{equation}
  (ut)^2 + (vt)^2 = 1.
\end{equation}
Solving we get $t = 1/\sqrt{u^2 + v^2},$ which we plug back into Equation~\ref{eq:r} and
get the cylindrical point
\begin{equation}
  \mathbf{p} = (x,y,z) = \frac{(u,v,w)}{\sqrt{u^2 + v^2}}.
\end{equation}
In cylindrical coordinates we have
\begin{eqnarray}
\mathbf{p} = \left(\theta, z\right), \ \ \ 
  \theta = \mbox{atan2}\left(y,x\right).
\end{eqnarray}
The resulting pixel coordinate in the input image is
\begin{equation}
(r,c) = \left(\frac{H}{2}\left(z + 1\right),\ \frac{W}{2\pi}\cdot(\theta + \pi)\right).
\end{equation}




\begin{verbatim}
for row = 0 .. outImage.H-1 {
    for col = 0 .. outImage.W-1 {
       s = 4.0*col/outImage.W
       face = floor(s)  // face = 0,1,2,3
       f = s - face     // f = frac(s)
       if face == 0 {   // (u,v,w) = point on cube
           u = 1
           v = 2*f - 1
       } else if face == 1 {
           u = 1 - 2*f
           v = 1
       } else if face == 2 {
           u = -1
           v = 1 - 2*f
       } else {
           u = 2*f - 1
           v = -1
       }
       w = 2.0*row/outImage.H - 1
       (x,y,z) = (u,v,w)/sqrt(u*u + v*v) // project onto cylinder
       theta = atan2(y,x)                // convert to cyl. coords
       r = inImage.H*(z + 1)/2.0         // map to input pixel
       c = inImage.W*(theta + pi)/(2*pi)
       outImage(row,col) = inImage.sample(r,c)
     }
}
\end{verbatim}


%\begin{tabbing}
%\end{tabbing}


\end{document}  